\begin{acknowledgements}

This dissertation would not have been possible without the contributions of many people. First and foremost, I would like to express my gratitude towards my inspirational advisor, Jianming Liang, for his continued guidance and support over the last five years. His motto ``simple, working, neat'' is a demonstration of his scientific profession and enthusiasm, which has also profoundly influenced and encouraged me to pursue an academic life. It has been a pleasure and a privilege to be mentored by Jianming, who teaches me how to think critically, present clearly, and conduct high-quality research. His understanding of which research directions will be impactful and where a project should move next are unmatched---it is thanks to his foresight that we finally made discoveries towards annotation-efficient deep learning in computer-aided diagnosis, which not only constitutes a major part of this dissertation but has also rewarded us with winning entries in competitions and best paper recognition from the research community. I sincerely appreciate Edward H. Shortliffe, Robert A. Greenes, Baoxin Li, and Murthy Devarakonda to serve on my dissertation committee and devote patience, time, and commitment to improving my dissertation and research. I would also like to acknowledge Hongkai Wang for introducing me to deep learning in 2015 before my Ph.D. journey.

A special thank you goes to my clinical partners, particularly Michael B. Gotway, for entrusting me with the pulmonary embolism project. Various works in this dissertation have collaborated closely with Michael---we not only published the corresponding methods in high-ranking conferences and journals but also investigated the clinical impact of computer-aided diagnosis, particularly for pulmonary embolism detection. I would also like to extend many thanks to Suryakanth R. Gurudu, R. Todd Hurst, and Michael G. Meyer, who provide valuable clinical datasets and extensive ground truths. Their contributions to making those deep learning methods possible in medical imaging were truly significant and irreplaceable.

A particularly warm thank you goes to my good friend and colleague Nima Tajbakhsh, with whom I have collaborated closely on two of the projects presented in this dissertation. His technical expertise and exceptional skill for scientific writing were pivotal for the success of these projects and publications. I also truly appreciate Jae Y. Shin for generously providing countless technical supports and dataset organizations. 

I want to thank all co-authors for their hard work to dedicate great publications, including Vatsal Sodha, Md Mahfuzur Rahman Siddiquee, Jiaxuan Pang, Ruibin Feng, and Lei Zhang. I also appreciate many wonderful colleagues for the verification of the algorithms and maintenance of open-source Github, including Fatemeh Haghighi, Mohammad Reza Hosseinzadeh Taher, Zuwei Guo, Pengfei Zhang, Shivam Bajpai. Specifically, it is an enjoyable time to have worked with Shivam Bajpai, who has adapted UNet++ and Models Genesis to the nnU-Net framework and won in the liver tumor segmentation competition. I feel so excited to share a memorable time with other students and colleagues at JLiang Lab in the last five years, including Nahid Islam, Douglas Amoo-Sargon, Dongao Ma, Qiufeng Wu, Diksha Goyal, Zijie Yuan, Naveen Sai Madiraju, Zac Winzurk, Shiv Gehlot, Winston T. Wang, Rujuta Panvalkar, Shailaja Sampat, Daniella Asare, and many others. 


Out of the lab, I also got much help from research fellows during the two amazing research internships: one at Mayo Clinic with Bradley J. Erickson, Panagiotis D. Korfiatis, Zeynettin Akkus, Mellissa S. Warner, Marius N. Stan, and the other at CHUM with An Tang, Milena Cerny, Lisa Di Jorio, Eugene Vorontsov, Emmanuel Montagnon. Besides, I really appreciate the effort of Fabian Isensee in providing the competitive nnU-Net framework and Pavel Yakubovskiy for providing well-organized segmentation models to the community.


I have benefited greatly from the ASU writing center and have two incredible tutors to thank, including Keerthi Shrikar Tatapudi and Alexis Pluhar, for proofreading this dissertation. 
Many thanks to the wonderful colleagues at ASU Skysound Innovations, including Spencer Hunter, Jessica Mandl, Angela Spencer, Patricia Stepp, Merissa R. Anderson, for the dedication of drafting and revising innumerable invention disclosures.

This dissertation has been supported partially by ASU and Mayo Clinic through a Seed Grant and an Innovation Grant, and partially by the National Institutes of Health (NIH) under Award Number R01HL128785. This dissertation has utilized the GPUs provided partially by the ASU Research Computing and partially by the Extreme Science and Engineering Discovery Environment (XSEDE) funded by the National Science Foundation (NSF) under grant number ACI-1548562.


Last but not least, I owe the greatest debt of gratitude to my parents, Wenlan Zhou and Lihua Gao, for their unreserved love, continued encouragement, and unconditional support to pursue my academic dreams. They are such a sweet audience for my research and presentation, even if they have no idea about a single word; they always listen until the end and contribute to the most views. 
I am also indebted to my girlfriend, Jessica Han, who has been nothing but supportive for years of company, particularly ``decorating'' every day during the COVID-19 pandemic.


\end{acknowledgements}