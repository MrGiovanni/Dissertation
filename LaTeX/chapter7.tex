\chapter{Conclusion}
\label{ch7}


Deep learning methods will empower many aspects of computer-aided diagnosis over the next decade, from medical image acquisition and interpretation to clinical decision making~\citep{esteva2019guide,zhou2021review}. Despite the expert human performance of deep learning methods in a few medical applications~\citep{gulshan2016development,esteva2017dermatologist,ardila2019end,mckinney2020international}, its prohibitively high annotation costs raise doubts about their feasibility of applying to those medical specialties that lack such magnitude of annotation. In this dissertation, we have systematically introduced our work in developing annotation-efficient deep learning that enables to (1) smartly identify the most significant subjects to be annotated, (2) effectively aggregate multi-scale image features to maximize the potential of existing annotations, and (3) directly extract medical knowledge from images without manual annotation. We have remarked our contributions in computer-aided diagnosis by supporting several aspects of medical image interpretation, including disease detection, classification, and segmentation. The experimental results on twelve distinct medical applications demonstrate that with a small part of the dataset annotated, we can deliver deep learning methods that match, or even outperform those that require annotating the entire dataset. This observation is encouraging and significant because it addresses the daunting challenge of limited annotated data---the main obstacle standing between deep learning methods and their clinical impact. Our devised methodologies are advantageous on over-represented diseases with abundant existing annotations and also shed new light on many more underrepresented diseases with the deep learning marvel, dramatically reducing annotation costs while maintaining high performance.

More importantly, we have been advocating open access, open data, and open source to benefit the research community. In our dissertation, eight out of the twelve medical applications were taken from publicly available medical imaging benchmarks (elaborated in Appendix~\ref{ap1}), ensuring the reproducibility of the results. Furthermore, we have released the codes and models to the public (detailed in Appendix~\ref{ap2}), making three developed techniques (ACFT, UNet++, and Models Genesis) open science to stimulate collaboration among the research community and to help translate these technologies to clinical practice. We first presented our ACFT, UNet++, and Models Genesis in CVPR~2017, DLMIA~2018, and MICCAI~2019, respectively. They have since been quickly adopted by the research community: reviewed by some of the most prestigious journals and conferences in the field, served as competitive baselines, and enlightened the development of more advanced approaches. Moreover, although our techniques were initially derived from the medical imaging context, their annotation-efficiency and generalizability have been demonstrated by independent research groups from alternative fields, such as text classification~\citep{oftedal2019uncertainty}, vehicle type recognition~\citep{huang2019cost}, streaming recommendation system~\citep{guo2019streaming}, image coloring~\citep{di2021color}, moon impact crater detection~\citep{jia2021moon}, microseismic monitoring~\citep{guo2021first}, etc.


Human annotation is one of the most significant cornerstones for algorithm development and evaluation. For the purpose of development, annotation-efficient deep learning facilitates quick, iterative improvements of the algorithm, whereas for performance evaluating, we still have to curate large, representative annotated datasets. In addition to the sufficient population of patients, we must also evaluate how the algorithms generalize to other medical images acquired from different devices, conditions, and sites---all of which must be annotated---before eventually adopting the techniques into clinical practice. Therefore, the increasing annotation demands are anticipated to continue troubling us with the lack of budget, time, and expertise. Big data is an inevitable trend in the future---with the increasing imaging studies, rising workloads of radiologists, and growing needs for technologies---we embrace the age of big data. The purpose of annotation-efficient deep learning is not to strangle the throat of annotating \textit{per se} but rather to speed up creating such datasets to enable high-performance deep learning methods with a minimal set of human expert annotation efforts. 

% Annotation is important for both training and validation.
% For training, annotation-efficient deep learning can reduce annotation cost.
% But for validation, annotation-efficient deep learning cannot.
% We still need many annotations for validation.
% Big data is inevitable in the future.
% Annotation-efficient is not to stop big data, but helping create big data with small human effort.

